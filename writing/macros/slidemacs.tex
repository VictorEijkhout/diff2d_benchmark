% -*- latex -*-
%%%%%%%%%%%%%%%%%%%%%%%%%%%%%%%%%%%%%%%%%%%%%%%%%%%%%%%%%%%%%%%%
%%%%%%%%%%%%%%%%%%%%%%%%%%%%%%%%%%%%%%%%%%%%%%%%%%%%%%%%%%%%%%%%
%%%%
%%%% This text file is part of the source of 
%%%% `Introduction to High-Performance Scientific Computing'
%%%% by Victor Eijkhout, copyright 2012-2023
%%%%
%%%% This book is distributed under a Creative Commons Attribution 3.0
%%%% Unported (CC BY 3.0) license and made possible by funding from
%%%% The Saylor Foundation \url{http://www.saylor.org}.
%%%%
%%%% slidemacs.tex : goes after snippetmacs
%%%%
%%%%%%%%%%%%%%%%%%%%%%%%%%%%%%%%%%%%%%%%%%%%%%%%%%%%%%%%%%%%%%%%
%%%%%%%%%%%%%%%%%%%%%%%%%%%%%%%%%%%%%%%%%%%%%%%%%%%%%%%%%%%%%%%%

%%
%% slide/book switches
%%
\newif\ifInBook\InBookfalse
\excludecomment{fullsources}
\let\slidebreak=\\
\newdimen\unitindent \unitindent=20pt

%% \input listingmacs
\lstdefinestyle{slidecode}{
  belowcaptionskip=1\baselineskip, frame=none, numbers=left,numberstyle=\tiny,
  breaklines=true, breakatwhitespace=true,
  postbreak=\hbox{{$\hookrightarrow$}}, % \textcolor{green}
  xleftmargin=.5\unitindent, showstringspaces=false,
  basicstyle=\codefontsize\ttfamily,
  keywordstyle=\bfseries\color{blue},
  commentstyle=\color{red!60!black},
  identifierstyle=\slshape\color{black},
  stringstyle=\color{green!60!black}, columns=fullflexible,
  keepspaces=true,tabsize=8,
  %% alsoletter={(),&*}
}
\lstset{style=slidecode}

%%%%
%%%% Outlining
%%%%
\usepackage{outliner}
\newcounter{sectionframe}
\OutlineLevelStart 0{\refstepcounter{sectionframe}
  \frame{\part{#1}\Large\bf #1}}
\OutlineLevelStart 1{\frame{\section*{#1}\large\bf #1}}
\OutlineLevelStart 2{\frame{\subsection*{#1}\large #1}}

%% \setcounter{tocdepth}{3}
%% \OutlineLevelStart 0 {
%%     \section{#1}
%%     \addcontentsline{toc}{section}{#1}
%%     \frame{\partpage}
%% }
%% %\OutlineLevelStart 1{\section{#1}}
%% \OutlineLevelStart 1{
%%   \subsection{#1}
%%   \addcontentsline{toc}{subsection}{#1}
%%   \begin{frame}\Large\bf#1
%%   \end{frame}
%% }
\def\sectionframe#{\Level 1 }
\def\subsectionframe#{\Level 2 }
\usepackage{framed}
\colorlet{shadecolor}{blue!15}
%% \OutlineLevelStart 2{\subsection{#1}
%%   \frame{\begin{shaded}\large #1\end{shaded}}}

%%
%% stuff
%%
\newcommand\furtherreading{\Level 0 {Further Reading}\label{sec:furtherreading-\chapshortname}}
\newcommand\heading[1]{\paragraph*{\textbf{#1}}}

{\catcode`\^^I=13 \globaldefs=1
 \newcommand\listing[2]{\begingroup\small\par\vspace{1ex}
  \catcode`\^^I=13 \def^^I{\leavevmode\hspace{40pt}}
  \noindent\fbox{#1}
  \verbatiminput{#2}\endgroup}
 \newcommand\codelisting[1]{\begingroup\small\par\vspace{1ex}
  \catcode`\^^I=13 \def^^I{\leavevmode\hspace{40pt}}
  \noindent\fbox{#1}
  \verbatiminput{#1}\endgroup}
}

%%
%% use beamer buttons for links to exercise slides
%%
\newcommand\exref[1]{%
  % split at the colon and attach a colon to catch the label of the exercise
  \exrefcolon#1:}
\def\exrefcolon ex:#1:{%
  % we use \label{ex:foo} and \frame[label=exfoo]
  \hyperlink{#1}{\noexpand\beamergotobutton{ex#1}}%
}
\newcommand\skeleton[1]{}

\newtheorem{remark}{Remark}
\expandafter\ifx\csname definition\endcsname\relax
    \newtheorem{definition}{Definition}
\fi
\expandafter\ifx\csname theorem\endcsname\relax
    \newtheorem{theorem}{Theorem}
\fi
\expandafter\ifx\csname lemma\endcsname\relax
    \newtheorem{lemma}{Lemma}
\fi

%%%%
%%%% Verbatim source handling
%%%%

% each chapter has a list of sources
\newtoks\chaptersourcelist
\newcommand\addchaptersource[1]{
  \edef\temp{\global\chaptersourcelist={\the\chaptersourcelist #1}}\temp
}
\newcommand\listchaptersources{
  \expandafter\ChapterSourceHeader\the\chaptersourcelist\LSR
  %\tracingmacros=2 \tracingonline=1
  %\texttt{\the\chaptersourcelist}\par
  \expandafter\ListSourcesRecursively\the\chaptersourcelist\LSR
}
\def\LSR{\LSR}
\def\ChapterSourceHeader#1\LSR{
  \def\test{#1\LSR}
  \ifx\test\LSR
  \else
    \Level 0 {Sources used in this chapter}
  \fi
}
\def\ListSourcesRecursively#1{
  \def\test{#1}
  \ifx\test\LSR
  \else
    % list the file
    \textbf{Listing of code #1}:
    {\footnotesize \verbatiminput{#1}}
    \par
    % continue
    \expandafter\ListSourcesRecursively
  \fi
}

%%%%
%%%% Prototypes
%%%%
\def\underscore{_}
\def\mpiRoutineRef#1{\RoutineRefStyle\verbatiminput{#1}}
\def\RoutineRefStyle{\scriptsize}
\newcommand\standardversion{}
\newcommand\mpiparmtextsize{1.3in}
\def\protoslide#{\bgroup \catcode`\_=12
  \afterassignment\protoslideinclude \def\protoname}
\def\protoslideinclude{%
  \begin{frame}[containsverbatim]\frametitle{\texttt\protoname}
      % mpi standard macros
      \def\MPI/{MPI}\def\mpi/{MPI}\def\RMA/{RMA}
      \def\mpifunc##1{\texttt{##1}}
      \let\mpiarg\mpifunc \let\mpicode\mpifunc
      \let\const\mpifunc  \let\constskip\mpifunc
    \tiny
    \edef\tmp{\lowercase{\def\noexpand\standardroutine{\protoname}}}\tmp
    \IfFileExists
        {standard/\standardroutine\standardversion.tex}
        {\input{standard/\standardroutine\standardversion}}
        {% if no standard file, then maybe handwritten file
          \IfFileExists
              {mpireference/\protoname.tex}
              {\verbatiminput{mpireference/\protoname}}{}}
        %\expandafter\mpiRoutineRef\expandafter{\protoname}
  \end{frame}\egroup
}
%%
%% MPL protos
%%
\def\mplprotoslide#{\bgroup \catcode`\_=12
  \afterassignment\mplprotoslideinclude \def\protoname}
\def\mplprotoslideinclude{%
  \begin{frame}[containsverbatim]\frametitle{\texttt{MPL: \protoname}}
      % doxy standard macros
    \def\hyperlink##1##2{##2}
    \def\href##1{}
    \def\hypertarget##1{}
    \let\+\relax
    \tiny
    \edef\tmp{\lowercase{\def\noexpand\standardroutine{\protoname}}}\tmp
    \IfFileExists
        {mplreference/\standardroutine.tex}
        {\lstinputlisting{mplreference/\standardroutine}}
        {Missing MPL proto: \standardroutine}
  \end{frame}\egroup
}
    
    
\usepackage{acronym}
\newwrite\acrowrite
\openout\acrowrite=acronyms.tex
\def\acroitem#1#2{\acrodef{#1}{#2}
    \write\acrowrite{\message{defining #1}\noexpand\acitem{#1}{#2}}
}
\acroitem{AVX}{Advanced Vector Extensions}
\acroitem{BSP}{Bulk Synchronous Parallel}
\acroitem{CAF}{Co-array Fortran}
\acroitem{CUDA}{Compute-Unified Device Architecture}
\acroitem{DAG}{Directed Acyclic Graph}
\acroitem{DSP}{Digital Signal Processing}
\acroitem{FPU}{Floating Point Unit}
\acroitem{FFT}{Fast Fourier Transform}
\acroitem{FSA}{Finite State Automaton}
\acroitem{GPU}{Graphics Processing Unit}
\acroitem{HPC}{High-Performance Computing}
\acroitem{HPF}{High Performance Fortran}
\acroitem{ICV}{Internal Control Variable}
\acroitem{MIC}{Many Integrated Cores}
\acroitem{MPMD}{Multiple Program Multiple Data}
\acroitem{MIMD}{Multiple Instruction Multiple Data}
\acroitem{MPI}{Message Passing Interface}
\acroitem{MTA}{Multi-Threaded Architecture}
\acroitem{NUMA}{Non-Uniform Memory Access}
\acroitem{OS}{Operating System}
\acroitem{PGAS}{Partitioned Global Address Space}
\acroitem{PDE}{Partial Diffential Equation}
\acroitem{PRAM}{Parallel Random Access Machine}
\acroitem{RDMA}{Remote Direct Memory Access}
\acroitem{RMA}{Remote Memory Access}
\acroitem{SAN}{Storage Area Network}
\acroitem{SaaS}{Software as-a Service}
\acroitem{SFC}{Space-Filling Curve}
\acroitem{SIMD}{Single Instruction Multiple Data}
\acroitem{SIMT}{Single Instruction Multiple Thread}
\acroitem{SM}{Streaming Multiprocessor}
\acroitem{SMP}{Symmetric Multi Processing}
\acroitem{SOR}{Successive Over-Relaxation}
\acroitem{SP}{Streaming Processor}
\acroitem{SPMD}{Single Program Multiple Data}
\acroitem{SPD}{symmetric positive definite}
\acroitem{SSE}{SIMD Streaming Extensions}
\acroitem{TLB}{Translation Look-aside Buffer}
\acroitem{UMA}{Uniform Memory Access}
\acroitem{UPC}{Unified Parallel C}
\acroitem{WAN}{Wide Area Network}
\acresetall
\closeout\acrowrite


\def\chaptertitle{\csname\chaptername title\endcsname}
\def\chaptershorttitle{\csname\chaptername shorttitle\endcsname}

%%%%
%%%% index macros without index
%%%%

\newif\ifShowRoutine
\def\indexmpishow#{\bgroup \InnocentChars
  \ShowRoutinetrue
  \afterassignment\mpitoindex\edef\indexedmpi}
\def\mpitoindex{%\tracingmacros=2
  \edef\tmp{%
    \noexpand\ifShowRoutine
        \noexpand\lstinline+\indexedmpi+\noexpand\nobreak
    \noexpand\fi
  }%
  \tmp
  \egroup\nobreak
}
\let\indexmpiref\indexmpishow
\let\indexmpidef\indexmpishow
\let\indexmpidepr\indexmpishow
\let\indexmpldef\indexmpishow
\let\indexmplref\indexmpishow
\let\indexmplshow\indexmpishow
\let\indexompshow\indexmpishow
\let\indexpetscshow\indexmpishow
\let\indexpragma\indexmpishow
\let\indexc\lstinline

\newenvironment
    {mplimpl}
    {\begin{quotation}\textsl{Implementation note:}\ }
    {\end{quotation}}

\newcommand\petscroutineslide[1]{
  \begin{frame}[containsverbatim]{#1}
    \footnotesize
    \verbatiminput{#1.tex}
  \end{frame}
}

%%%%
%%%% Environments
%%%%
\newcounter{slidecount}
\setcounter{slidecount}{0}
%% what is the optional argument for?
\newenvironment
    {numberedframe}[2][]
    {%% if there is an optional tag, reuse the previous slide number
      \def\empty{}\def\opt{#1}
      \ifx\empty\opt
          \refstepcounter{slidecount}
      \else
          \addtocounter{framenumber}{-1}    
      \fi
      %% \arabic{slidecount}
      \begin{frame}[containsverbatim]{#2}
    }
    {\end{frame}}
\newenvironment{question}{\begin{quotation}\textbf{Question.\ }}{\end{quotation}}
\newenvironment{fortrannote}
  {\begin{quotation}\noindent\textsl{Fortran note.\kern1em}\ignorespaces}
  {\end{quotation}}
\newenvironment{pythonnote}
  {\begin{quotation}\noindent\textsl{Python note.\kern1em}\ignorespaces}
  {\end{quotation}}
\newenvironment{taccnote}
  {\begin{quotation}\noindent\textsl{TACC note.\kern1em}\ignorespaces}
  {\end{quotation}}
\newenvironment{mpifour}
  {\begin{quotation}\textbf{MPI-4:\ }\ignorespaces}{\end{quotation}}

