% -*- latex -*-
%%%%%%%%%%%%%%%%%%%%%%%%%%%%%%%%%%%%%%%%%%%%%%%%%%%%%%%%%%%%%%%%
%%%%%%%%%%%%%%%%%%%%%%%%%%%%%%%%%%%%%%%%%%%%%%%%%%%%%%%%%%%%%%%%
%%%%
%%%% This text file is part of the source of 
%%%% `The Art of HPC, vol 1: The Science of Computing'
%%%% by Victor Eijkhout, copyright 2012-2024
%%%%
%%%% commonmacs : very common macros
%%%%
%%%%%%%%%%%%%%%%%%%%%%%%%%%%%%%%%%%%%%%%%%%%%%%%%%%%%%%%%%%%%%%%
%%%%%%%%%%%%%%%%%%%%%%%%%%%%%%%%%%%%%%%%%%%%%%%%%%%%%%%%%%%%%%%%

% regular formatting
\newcommand\innocentcharacters{%
  \catcode`\_=12 \catcode`\#=12
  \catcode`\>=12 \catcode`\<=12
  \catcode`\&=12 \catcode`\^=12
  \catcode`\$=12 \catcode`\~=12
}
\def\n#{\bgroup
  \innocentcharacters
  \def\\{\char`\\}\relax
  \tt \let\next=}
\let\InnocentChars\innocentcharacters

\newcommand\makeusletter{\catcode`\_=11 }
\newcommand\makeusother{\catcode`\_=12 }

\usepackage{listings,verbatim,xcolor}% needed for parsing some aux files
\usepackage{booktabs}

\newcommand\inv{^{-1}}
\newcommand\invt{^{-t}}
\newcommand\setspan[1]{[\![#1]\!]}
\newcommand\fp[2]{#1\cdot10^{#2}}
\newcommand\fxp[2]{\langle #1,#2\rangle}

\newcommand\diag{\mathop{\mathrm {diag}}}
\newcommand\argmin{\mathop{\mathrm {argmin}}}
\newcommand\defined{
  \mathrel{\lower 5pt \hbox{${\equiv\atop\mathrm{\scriptstyle D}}$}}}
\newcommand\lulubreak{\message{Hard page break!}\pagebreak\relax}

\newcommand\bbP{\mathbb{P}}
\newcommand\bbR{\mathbb{R}}

%%%%
%%%% input file without header
%%%%
\newcommand\strippedinput[2]{
  \immediate\write18{
    cd #1 && ls #2 
    && awk 'p==1 && !/codesnippet/ {print} NF==0 {p=1}' #2 > #2.stripped.out
  }
  \IfFileExists
      {#1/#2.stripped.out}
      {\lstinputlisting{#1/#2.stripped.out}}
      {\message{Could not strip file #1/#2}
        \lstset{basicstyle=\tiny\ttfamily}
        \lstinputlisting{#1/#2}}
}
%% something similar
\newcommand\ListStrippedSource[2]{
  \begingroup \footnotesize
  \IfFileExists{#1/#2.stripped}
  {}
  {\immediate \write 18
    { cd #1 ; /Users/eijkhout/Current/istc/scientific-computing-private/stripsource #2 }}
  \message{Inputting stripped source: #1/#2.stripped}
  \verbatiminput{#1/#2.stripped}
  
  \endgroup
}

%%
%% title page
%%
\newcommand\formattime{%
  \begingroup
  \count0=\time \divide\count0 by 60
  \count1=\count0 \multiply\count1 by 60
  \advance\time by -\count1 
  last formatted \today, \the\count0\relax:\the\time\relax
  \endgroup}

%%
%% cmake
%%
\begin{packt}
\newcommand\cmakeExampleFiles[1]{}
\end{packt}
\begin{nopackt}
\newcommand\cmakeExampleFiles[1]{%
  (The files for this example are in \n{#1}.)
}
\end{nopackt}
