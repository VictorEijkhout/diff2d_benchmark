% -*- latex -*-
%%%%%%%%%%%%%%%%%%%%%%%%%%%%%%%%%%%%%%%%%%%%%%%%%%%%%%%%%%%%%%%%
%%%%%%%%%%%%%%%%%%%%%%%%%%%%%%%%%%%%%%%%%%%%%%%%%%%%%%%%%%%%%%%%
%%%%
%%%% This text file is part of the source of 
%%%% `Parallel Computing'
%%%% by Victor Eijkhout, copyright 2012-2024
%%%%
%%%% cppcon2024slides.tex : master file for an MPI course
%%%%
%%%%%%%%%%%%%%%%%%%%%%%%%%%%%%%%%%%%%%%%%%%%%%%%%%%%%%%%%%%%%%%%
%%%%%%%%%%%%%%%%%%%%%%%%%%%%%%%%%%%%%%%%%%%%%%%%%%%%%%%%%%%%%%%%

\documentclass[10pt]{beamer}

\input courseformat

\usepackage{geometry,fancyhdr,wrapfig}
\usepackage{amssymb,amsmath,verbatim,graphicx,pslatex,multicol}
\usepackage{comment}
\newif\ifInBook \InBooktrue
\input book.inex
\input acromacs
\def\latexengine{}
%% \input bookmacs
%% \input articlemacs
%% \input ../../../scientific-computing-private/macros/commonmacs
%% \input idxmacs
%% \input idxpkgmacs
\input listingmacs
%% \input snippetmacs
\input ../../../scientific-computing-private/macros/tikzplot
\excludecomment{packt}

\lstset{ keywordstyle=\usebeamercolor*[fg]{palette primary} }
\newcommand\codesnippetsdir{../snippets}

%%
%% refer to sections in the HPC book
%%
\usepackage{xr-hyper}
% vol 1
\externaldocument[HPSC-]{scicompbook}
%%\newcommand\HPSCref[2][section]{HPC book~\cite{ISTCwebpage}, #1~\ref{HPSC-#2}}
%% \input macroload

\def\qrcode{}
\begin{document}
\input lang

\author[Eijkhout]{Victor Eijkhout}
\date[CppCon2024]{CppCon 2024}
%% \normalsize last formatted \today}
\title[C++ Parallel]{Modern C++ for Parallelism\\ in Scientific Computing}
\maketitle

\begin{frame}{Scientific computing parallelism}
  \begin{itemize}
  \item Large amounts of data: \\
    often cartesian multi-dimensional arrays, sometimes unstructured data
  \item Large amounts of parallelism:\\
    each element of output array independent.
  \item No explicit threading\\
    parallelism created by some runtime
  \item Range algorithm notion:\\
    do some operation on each element of a dataset
  \end{itemize}
\end{frame}

\begin{frame}[containsverbatim]{Power method}
  \begin{quote}
    \begin{tabbing}
      Let $A$ a matrix of interest\\
      Let $x$ be a random vector\\
      For \=iterations until convergence\\
      \> compute the product $y\leftarrow Ax$\\
      \> compute the norm $\gamma=\| y \|$\\
      \> normalize $x\leftarrow y/\gamma$\\
    \end{tabbing}
  \end{quote}
  \begin{itemize}
  \item Method for computing largest eigenvalue of a matrix
  \item Also Google Pagerank
  \item Stands for many scientific codes: Krylov methods, eigenvalues
  \end{itemize}
\end{frame}

\begin{frame}[containsverbatim]{Array parallelism}
  Traditional C implementation:
  
  \cxxverbatimsnippet{d2dscaleseq}
  \cxxverbatimsnippet{d2dseqindex}
  \begin{itemize}
  \item Two~/ three-dimensional loop
  \item all dimensions large
  \item every output element independent
  \end{itemize}
\end{frame}

\begin{frame}[containsverbatim]{Reductions}
  $\ell_2$ reduction:%
  \cxxverbatimsnippet{d2dnormseq}
  \begin{itemize}
  \item Parallel except for the accumulation
  \item Obviously should not be done through atomic operation
  \end{itemize}
\end{frame}

\begin{frame}[containsverbatim]{Stencil computation}
  \cxxverbatimsnippet{d2d5ptseq}
  \begin{itemize}
  \item Differential operator~/ image convolution
  \item Structure can be more complicated in scientific codes
  \end{itemize}
\end{frame}

\begin{frame}[containsverbatim]{OpenMP parallelism}
  Annotate loops as parallel and/or reduction:
  \cxxverbatimsnippet{d2dnormomp}
  \begin{itemize}
  \item Static assigment of iterations to threads by default
  \item Highly controlled affinity
  \item `\n{oned}' as above, `\n{clps}' for both loops collapsed
  \end{itemize}
\end{frame}

\begin{frame}[containsverbatim]{\texttt{mdspan} and \texttt{cartesian\_product}}
  \hbox\bgroup
  \cxxverbatimsnippet{d2dspan0}
  \cxxverbatimsnippet{d2dspan2}
  \egroup

  \cxxverbatimsnippet{d2dinner}
  %% \begin{itemize}
  %% \item 
  %% \end{itemize}
\end{frame}

\begin{frame}[containsverbatim]{}
\cxxverbatimsnippet{d2d5ptspan}
  %% \begin{itemize}
  %% \item 
  %% \end{itemize}
\end{frame}

\begin{frame}[containsverbatim]{Sycl}
  Open standard, but mostly pushed by Intel
  \cxxverbatimsnippet{syclbufaccess}
  \begin{itemize}
  \item Heterogeneous CPU/GPU code,\\
    transparent data movement
  \item Range algorithm-like syntax,\\
    but explicit task queue
  \end{itemize}
\end{frame}

\begin{frame}[containsverbatim]{Kokkos}
  Open Source heterogeneous execution layer
  \cxxverbatimsnippet{kokkosbufaccess}
  \begin{itemize}
  \item Implicit task queue
  \item Two-dimensional indexing
  \item Range algorithm-like philosophy
  \end{itemize}
\end{frame}

\begin{frame}[containsverbatim]{Comparing models (Intel)}
  \input d2dimodeltime

  Intel compiler. C-style variant fastest.
\end{frame}

\begin{frame}[containsverbatim]{Ratio to fastest (Intel)}
  \input d2dimodelratio
\end{frame}

\begin{frame}[containsverbatim]{Comparing models (Gcc)}
  \input d2dgmodeltime

  Gcc compiler. less variance between variants
\end{frame}

\begin{frame}[containsverbatim]{Ratio to fastest (Gcc)}
  \input d2dgmodelratio
\end{frame}

\begin{frame}[containsverbatim]{Can we use execution policies?}
  \begin{itemize}
  \item Last time I tried there was a compiler issue
  \item That 5-point stencil is hard to express in range views!
  \end{itemize}
\end{frame}

\end{document}

\begin{frame}[containsverbatim]{}
  \begin{itemize}
  \item 
  \end{itemize}
\end{frame}

\begin{frame}[containsverbatim]{}
  \begin{itemize}
  \item 
  \end{itemize}
\end{frame}

\begin{frame}[containsverbatim]{}
  \begin{itemize}
  \item 
  \end{itemize}
\end{frame}


