% -*- latex -*-
%%%%%%%%%%%%%%%%%%%%%%%%%%%%%%%%%%%%%%%%%%%%%%%%%%%%%%%%%%%%%%%%
%%%%%%%%%%%%%%%%%%%%%%%%%%%%%%%%%%%%%%%%%%%%%%%%%%%%%%%%%%%%%%%%
%%%%
%%%% This text file is part of the source of 
%%%% the diff2d benchmark for parallel stencil codes
%%%% by Victor Eijkhout, copyright 2012-2025
%%%%
%%%% diff2d.tex : driver file for the writeup
%%%%
%%%%%%%%%%%%%%%%%%%%%%%%%%%%%%%%%%%%%%%%%%%%%%%%%%%%%%%%%%%%%%%%
%%%%%%%%%%%%%%%%%%%%%%%%%%%%%%%%%%%%%%%%%%%%%%%%%%%%%%%%%%%%%%%%

\documentclass[11pt,fleqn]{artikel3}

\input macroload

\begin{document}
\title{Performance analysis of a stencil power method}
\author{Victor Eijkhout\thanks{{\tt
      eijkhout@tacc.utexas.edu}, Texas Advanced Computing Center, The
    University of Texas at Austin.
  }, Yojan Chitkara, Daksh Chaplot}
\maketitle

\part{Writeup}

\input d2dtests

\paragraph*{Acknowledgement}
This work was supported by
the Intel OneAPI Center of Excellence, and the 
TACC STAR Scholars program,
funded by generous gifts from TACC industry partners, including Intel, Shell, Exxon, and Chevron.

\part{Appendix}

\Level 0 {Run scripts, settings, data}

Compile:
\begin{lstlisting}
## delete all binaries
make clean
## rebuild all
make cmake
\end{lstlisting}
You will get a listing of all binaries that have been built
into the \lstinline{bin} directory.
Some versions may be missing for perfectly legitimate reasons,
such as sycl for gcc.

The sofware relies on \lstinline{cxxopts}, \lstinline{mdspan}, \lstinline{kokkos}.
You can install those yourself, or let cmake download~/ install them for you.

Comparison of programming models:
\begin{lstlisting}
./compare_models.sh -c spr -q spr-dev -p 112 -s -t -g all
\end{lstlisting}
where
\begin{itemize}
\item \n{-c spr} names the \lstinline{spr} processor; this is purely for naming purposes.
\end{itemize}
This script is heavily SLURM-based.
\begin{itemize}
\item \n{-s} prepends \lstinline{srun} with various parameters to the run command;
  the alternative is to run on a compute node.
\item \n{-q spr-dev} names the SLURM queue if \lstinline{srun} is used.
\item \n{-p 112} gives the core count for the node if \lstinline{srun} is used;
  otherwise \lstinline{SLURM} parameters are queried.
\end{itemize}
The actual program run is done by another script that does a scaling study.
As a result, a file with extension \lstinline{.out} is generated.
\begin{itemize}
\item \n{-t} prints out trace information.
\item \n{-g} indicates that the \lstinline{.out} file 1.~is renamed to \lstinline{.runout},
  and 2.~is added to the repository. This of course requires you to have a writable fork.
\end{itemize}

\Level 0 {Data}

Table~\ref{fig:diff2d-variants} left.

\lstinputlisting{plots/spr-models-intel.csv}

Right.

\lstinputlisting{plots/spr-models-intel-sp.csv}

Table~\ref{fig:diff2d-variants} left.

\lstinputlisting{plots/spr-models-gcc.csv}

Right.

\lstinputlisting{plots/spr-models-gcc-sp.csv}

Table~\ref{fig:diff2d-procs} left.

skx

\lstinputlisting{plots/intel-skx.csv}

csx

\lstinputlisting{plots/intel-csx.csv}

icx

\lstinputlisting{plots/intel-icx.csv}

spr

\lstinputlisting{plots/intel-spr.csv}


\end{document}
